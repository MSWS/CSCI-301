\documentclass{article}

\usepackage[tmargin=0.5in,bmargin=0.25in]{geometry}
\usepackage{amsmath, amssymb, amsthm}
\usepackage{enumitem}
\usepackage{makecell}
% \usepackage{lscape}

\title{\vspace{-5ex}CSCI 301 HW 4}
\author{Isaac Boaz}

\newtheorem*{theorem}{Proposition}

\begin{document}
\maketitle

\section*{Problem 1}
Define a relation \(R\) on \(\mathbb{Z}\) as \(xRy\) if and only if \(3 \mid (2x + y)\)

\begin{enumerate}[label=\alph*.]
    \item Prove that \(R\) is an equivalence relation. \\
          To show that \(R\) is an equivalence relation, we must show that \(R\) is reflexive, symmetric, and transitive. \\

          \textbf{Reflexive}
          \(R\) is reflexive if and only if \(xRx\) for all \(x \in \mathbb{Z}\). \\
          That is to say, \(R\) is reflexive if and only if \(3 \mid (2x + x)\) for all \(x \in \mathbb{Z}\).
          \begin{proof}
              By contradiction.

              Suppose \(3 \nmid (2x + x)\) \\
              We can rewrite \(2x + x\) as \(3x\). \\
              By definition of divisibility, \(3 \mid 3x\). \\
              Thus \(2x + x\) is both divisible by \(3\) and not divisible by \(3\). Contradiction! \\
              Therefore \(3 | 2x + x\)
          \end{proof}

          \textbf{Symmetric}
          \(R\) is symmetric if and only if \(xRy \implies yRx\) for all \(x, y \in \mathbb{Z}\). \\
          That is to say, \(R\) is symmetric if and only if \(3 \mid (2x + y) \implies 3 \mid (2y + x)\) for all \(x, y \in \mathbb{Z}\).
          \begin{proof}
              By definition of divisibility, we can write \(2x + y\) as
              \begin{align*}
                  2x + y = 3a, \space a \in \mathbb{Z} \\
                  y = 3a - 2x
              \end{align*}
              Thus
              \begin{align*}
                  2y + x & = 2(3a - 2x) + x               \\
                         & = 6a - 4x + x                  \\
                         & = 6a - 3x                      \\
                         & = 3(2a - x)                    \\
                         & = 3b \text{ where } b = 2a - x
              \end{align*}
              Therefore by definition of divisibility, \(3 \mid (2y + x)\)
          \end{proof}
          \textbf{Transitive} \(R\) is transitive if and only if \(xRy \land yRz \implies xRz\) for all \(x, y, z \in \mathbb{Z}\). \\
          That is to say, \(R\) is transitive if and only if \(3 \mid (2x + y) \land 3 \mid (2y + z) \implies 3 \mid (2x + z)\)
          \begin{proof}
              By definition of divisibility, we can rewrite the first two statements as
              \begin{align*}
                  2x + y = 3a, \space a \in \mathbb{Z} \\
                  2y + z = 3b, \space b \in \mathbb{Z}
              \end{align*}
              \begin{align*}
                  y = 3a - 2x \\
                  z = 3b - 2y
              \end{align*}
              which allows us to define
              \begin{align*}
                  z & = 3b - 2y         \\
                    & = 3b - 2(3a - 2x) \\
                    & = 3b - 6a + 4x    \\
              \end{align*}
              Plugging this in to \(2x + z\) we get
              \begin{align*}
                  2x + z & = 2x + 3b - 6a + 4x                 \\
                         & = 6x + 3b - 6a                      \\
                         & = 3(2x + b - 2a)                    \\
                         & = 3c \text{ where } c = 2x + b - 2a
              \end{align*}
              Therefore by definition of divisibility, \(3 \mid (2x + z)\)
          \end{proof}
    \item Describe the equivalence classes of \(R\). \\
          To determine the equivalence classes of \(R\), we must observe that there are three possible values that result from \(3 \mid 2x + y\): \(0, 1, 2\). \\
          We can rewrite this equation as \(y \equiv -2x \mod 3\). \\
          Therefore, the set of all integers \(y\) that satisfies \(y \equiv -2x \mod 3\) is the equivalence class \([x]\): \\
          \begin{equation*}
              [x] = \{y \in \mathbf{Z} : y \equiv -2x \mod 3\}
          \end{equation*}
          \([0]_R = \{\dots, -6, -3, 0, 3, 6, \dots\}\) \\
          \([1]_R = \{\dots, -5, -2, 1, 4, 7, \dots\}\) \\
          \([2]_R = \{\dots, -4, -1, 2, 5, 8, \dots\}\) \\
\end{enumerate}

\pagebreak

\section*{Problem 2}
Prove the function \(f:\mathbb{Z} \times \mathbb{Z} \rightarrow \mathbb{Z} \times \mathbb{Z}\) defined by the formula \(f(m, n) = (5m + 4n, 4m + 3n)\) is bijective. Find its inverse \(f^{-1}\).

% \begin{proof}
To prove that \(f\) is bijective, we must show that \(f\) is both injective and surjective. \\

\textbf{Injective} \(f\) is injective if and only if \(f(a) = f(b) \implies a = b\) for all \(a, b \in \mathbb{Z} \times \mathbb{Z}\). \\
That is to say, \(f\) is injective if and only if \(f(m, n) = f(p, q) \implies (m, n) = (p, q)\) for all \(m, n, p, q \in \mathbb{Z}\).
\begin{proof}
    Suppose \(f(a, b) = f(c, d)\). \\
    Then
    \begin{align*}
        5a + 4b = 5c + 4d \\
        4a + 3b = 4c + 3d
    \end{align*}
    \begin{align*}
        3b                                 & = 4c + 3d - 4a                            \\
        b                                  & = \frac{4c+3d-4a}{3} = \frac{4}{3}(c-a)+d \\
        5a + 4(\frac{4}{3}(c-a)+d)         & = 5c + 4d                                 \\
        5a + \frac{16}{3}(c-a) + 4d        & = 5c + 4d                                 \\
        5a + \frac{16c}{3} - \frac{16a}{3} & = 5c                                      \\
        -\frac{a}{3} + \frac{16c}{3}       & = 5c                                      \\
        -\frac{a}{3}                       & = -\frac{c}{3}                            \\
        a                                  & = c
    \end{align*}
    \begin{align*}
        4a                                 & = 4c + 3d - 3b           \\
        a                                  & = c + \frac{3}{4}(d - b) \\
        5(c+\frac{3}{4}(d-b)) + 4b         & = 5c + 4d                \\
        \frac{15}{4}(d-b) + 4b             & = 4d                     \\
        \frac{15}{4}d - \frac{15}{4}b + 4b & = 4d                     \\
        \frac{b}{4}                        & = \frac{d}{4}            \\
        b                                  & = d
    \end{align*}
    Thus, \(a = c\) and \(b = d\). \\
    Therefore \(f(a, b) = f(c, d) \implies (a, b) = (c, d)\)
\end{proof}

\textbf{Surjective} \(f\) is surjective if and only if \(f(\mathbb{Z} \times \mathbb{Z}) = \mathbb{Z} \times \mathbb{Z}\). \\
That is to say, \(f\) is surjective if and only if \(\forall (m, n) \in \mathbb{Z} \times \mathbb{Z}, \exists (a, b) \in \mathbb{Z} \times \mathbb{Z} \text{ such that } f(a, b) = (m, n)\).

\end{document}