\documentclass{article}

\usepackage[tmargin=0.5in,bmargin=0.25in,lmargin=0.4in]{geometry}
\usepackage{amsmath, amssymb, amsthm}
\usepackage{enumitem}
\usepackage{makecell}
\usepackage{lscape}

\title{\vspace{-5ex}CSCI 301 HW 11}
\author{Isaac Boaz}

\renewcommand\theadalign{bc}
\renewcommand\theadfont{\bfseries}
\renewcommand\theadgape{\Gape[4pt]}
\renewcommand\cellgape{\Gape[4pt]}

\begin{document}
\maketitle

\section*{Problem 1}
Let
\begin{align*}
   R &= \text{Rock} & = 150 \\
   C &= \text{Classical} & = 100 \\
   J &= \text{Jazz} & = 75
\end{align*}
\begin{align*}
    R \wedge C = 30 \\
    R \wedge J = 20 \\
    C \wedge J = 10 \\
    R \wedge C \wedge J = 5
\end{align*}

\begin{enumerate}[label=\alph*]
    \item \begin{align*}
        \text{Rock Music Only} = R = 150 \\
        \text{Classical Music Only} = C = 100 \\
        \text{Jazz Music Only} = J = 75
    \end{align*}
    \item \begin{align*}
        \text{Rock and Classical} = R \wedge C = 30 \\
        \text{Rock and Jazz} = R \wedge J = 20 \\
        \text{Classical and Jazz} = C \wedge J = 10 \\
    \end{align*}
\end{enumerate}

\pagebreak

\begin{landscape}
    \section*{Problem 2}
    \footnotesize
    \begin{tabular}{|c|c|c|c|c|c|c|c|}
        \hline
        \makecell{Alice committed\\the burglary} &
        \makecell{Bob committed\\the burglary} &
        \makecell{Carol comitted\\the burglary} &
        \makecell{The burglar was\\wearing gloves} &
        \makecell{The burglar was\\wearing a hat} &
        \makecell{If the burglar was wearing gloves,\\ then it was not Bob.} &
        \makecell{The thief was wearing a hat,\\ only if it was not Carol.} &
        \makecell{If the burglar was not wearing gloves,\\ then it was not Alice.} \\
        \hline
        A & B & C & G & H & \(G \implies \tilde{B}\) & \(\tilde{C} \implies H\) & \(\tilde{G} \implies \tilde{A}\) \\
        \hline
        F & T & T & F & F & T & T & T \\
        F & T & T & F & T & T & T & T \\
        F & T & T & T & F & F & T & T \\
        F & T & T & T & T & F & T & T \\
        T & F & T & F & F & T & T & F \\
        T & F & T & F & T & T & T & F \\
        T & F & T & T & F & T & T & T \\
        T & F & T & T & T & T & T & T \\
        T & T & F & F & F & T & F & F \\
        T & T & F & F & T & T & T & F \\
        T & T & F & T & F & F & F & T \\
        T & T & F & T & T & F & T & T \\
        \hline
        T & T & F & F & F & T & F & F \\
        \textbf{T} & \textbf{T} & \textbf{F} & \textbf{T} & \textbf{F} & \textbf{F} & \textbf{F} & \textbf{T} \\
        \hline
    \end{tabular}\\
    \normalsize
    \subsection*{Steps Taken}
    \begin{enumerate}[label=\arabic*.]
        \item I first generated all possible 5-long combinations of Truth and Falsehood.
        \item Since we are told there are two burglars, I filtered any combination that did not have two Truths in the first three positions.
        \item I then did the logic for each potential combination.
        \item Since we are told that only the innocent person told the truth (ie only one statement should be true), I filtered combinations that had more than 1 truthful statement.
        \item This leaves us with the bottom two possible combinations.
        \item Lastly, since we know the innocent person told the truth, I cross-referenced which person had the truthful statement with which person is marked innocent.
        \item This gives us the conclusion that Alice and Bob are the burglars and that the burglar was wearing gloves (but not a hat).
    \end{enumerate}
\end{landscape}

\pagebreak

\section*{Problem 3}
\begin{enumerate}[label=\arabic*)]
    \item \begin{enumerate}
        \item \(\forall a,b \in \mathbf{Z}, E(ab) \land E(a+b) \implies E(a) \land E(b)\)
        % \item For some a,b in Z. If either ab or a+b are not even, then both a and b are not even.
        \item \(\exists a,b \in \mathbf{Z}, \neg (E(ab) \lor E(a+b)) \implies \neg (E(a) \lor E(b))\)
        \item \((E(ab) \lor E(a+b)) \lor \neg (E(a) \lor E(b))\)
    \end{enumerate}
    \item \begin{enumerate}
        \item \(\forall a,b \in Z, 4 \mid (a^2 + b^2) \implies \neg O(a) \land \neg O(b)\)
        \item \(\exists a,b \in Z, \neg(4 \mid (a^2 + b^2)) \implies O(a) \lor O(b)\)
        \item \(\exists a,b \in Z, (4 \mid (a^2 + b^2)) \lor O(a) \lor O(b)\)
    \end{enumerate}
\end{enumerate}

\end{document}
