\documentclass{article}

\usepackage[tmargin=0.5in,bmargin=0.25in,lmargin=0.4in]{geometry}
\usepackage{amsmath, amssymb, amsthm}
\usepackage{enumitem}
\usepackage{makecell}
\usepackage{lscape}

\title{\vspace{-5ex}CSCI 301 HW 2}
\author{Isaac Boaz}

\newtheorem*{theorem}{Proposition}

\begin{document}
\maketitle

\begin{theorem}
    Let \(a, b \in \mathcal{Z}\). \((a + 5)b^2 \in E\) if and only if \(a \in O \lor b \in E\).
\end{theorem}

\begin{proof}
    Let's first show that \((a + 5)b^2 \in E \implies a \in O \lor b \in E\) by contraposition. \\
    Suppose \(a \notin O \land b \notin E\). \\
    Thus \(a\) is even and \(b\) is odd. \\
    By definition of even and odd, \(a = 2n_1, b = 2n_2 + 1\) where \(n_1, n_2 \in \mathcal{Z}\). \\
    Thus \((a + 5)b^2 = (2n_1 + 5)(2n_2+1)^2\) \\
    \(= 8 n_1 n_2^2 + 20 n_2^2 + 8 n_1 n_2 + 20 n_2 + 2 n_1 + 5\) \\
    \(= 2z + 1\) where \(z = 4 n_1 n_2^2 + 10 n_2^2 + 4 n_1 n_2 + 10 n_2 + n_1 + 2 \in \mathcal{Z}\) \\
    Therefore by definition of odd \((a + 5)b^2\) is not even. \\
% \end{proof}

% \begin{proof}
%     Let's now show that \(a \in O \lor b \in E \implies (a + 5)b^2 \in E\). \\
    \noindent
    Conversely,
    Suppose \(a \in O \lor b \in E\). \\
    Case 1: \(a \in O\) \\
    \indent
    Then by definition of odd \(a = 2n + 1\) where \(n \in \mathcal{Z}\). \\
    \indent
    Thus \((a + 5)b^2 = (2n + 6)b^2\) \\
    \indent
    \(= 2b^2(n + 3) = 2z\) where \(z = b^2(n + 3) \in \mathcal{Z}\) \\
    \indent
    Therefore by definition of even \((a + 5)b^2\) is even. \\
    Case 2: \(b \in E\) \\
    \indent
    Then by definition of even \(b = 2n\) where \(n \in \mathcal{Z}\). \\
    \indent
    Thus \((a + 5)b^2 = (a + 5)(2n)^2\) \\
    \indent
    \(= 4n^2(a+5) = 2z\) where \(z = 2n^2(a+5) \in \mathcal{Z}\) \\
    \indent
    Therefore by definition of even \((a + 5)b^2\) is even. \\
    Therefore \((a + 5)b^2 \in E\). \\

    \noindent
    Therefore \((a + 5)b^2 \in E \iff a \in O \lor b \in E\). \\
\end{proof}

\pagebreak

\begin{theorem}
    \(\forall n \in \mathcal{Z}, 3 \nmid (n^2 - 5)\)
    \begin{proof}
        By contradiction. \\
        Suppose \(\exists n \in \mathcal{Z} \text{ s.t. } 3 \mid (n^2 - 5)\) \\
        By definition of divisibility, \((n^2 - 5) = 3z\), where \(z \in \mathcal{Z}\) \\
        Solving for \(n^2\) we get \(n^2 = 3z + 5\) \\
        When dividing by 3, there are 3 possible remainders (0, 1, 2). \\
        Suppose \(m \in \mathcal{Z}\) \\
        Case 1: \(n = 3m\) \\
        \indent
        Then \(n^2 = (3m)^2 = 9m^2 = 3(3m^2)\) which has a remainder of 0. \\
        Case 2: \(n = 3m + 1\) \\
        \indent
        Then \(n^2 = (3m + 1)^2 = 9m^2 + 6m + 1 = 3(3m^2 + 2m) + 1\) which has a remainder of 1. \\
        Case 3: \(n = 3m + 2\) \\
        Then \(n^2 = (3m + 2)^2 = 9m^2 + 12m + 4 = 3(3m^2 + 4m) + 4\) which has a remainder of 4. \\
        Thus \(n^2\) can only have remainders of 0, 1, or 4 when divided by 3. \\
        However, \(n^2 = 3z + 5\) can only have a remainder of 2. \\
        Therefore \(n^2 = 3z + 5\) cannot be divisible by 3. \\
    \end{proof}
\end{theorem}

\end{document}
